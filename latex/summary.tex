% Summary
\chapter{Concluding Remarks} \label{ch:conclusion}
\begin{comment}

This work develops a shared control framework between adaptive autopilots and human operators of aerial vehicles. The autonomous control design (autopilot) builds on two recent advances in adaptive control theory, namely the use of closed-loop reference models for improved transient performance, and computationally efficient control designs for output-feedback systems having relative degree two or greater. The targeted role of the human operator is motivated by unmanned aerial platforms remotely supervised by humans (``human-on-the-loop''), and the limitations that come with remote manual control. Under our shared control framework, the human operator and adaptive autopilot designs form a shared response to dynamical anomalies. The shared control response is demonstrated in simulation on the longitudinal dynamics of an unmanned HALE VFA model. 

% TODO merge above (CPHS) with below (GNC)

When human pilots are suddenly presented with unfamiliar dynamics, they attempt to adapt their feedback gains, but performance deteriorates which increases the potential for loss of control events. Autopilots based on adaptive control algorithms present the potential to remove the human pilot from low-level control tasks and instead allow them to make high-level decisions, such as the switching of the controller structure to a suitable dimension which allows the adaptive autopilot to retain autonomous control of the vehicle. The human pilot collaborates with the autopilot in the detection and diagnosis of the anomaly and relegates corrective actions to an adaptive flight control system. As an autonomous adaptive flight control system will be able to control these low-level tasks with ease, such a shared controller may prove to be more advantageous in the face of severe anomalies.

We explore the behavior of a shared controller in the context of the roll dynamics of an aircraft, in the face of two different kinds of anomalies. In both cases, the pilot's task is to perceive if there is a change in the order of the vehicle dynamics, and convey this change to the adaptive autopilot. The resulting shared control action was shown to lead to satisfactory performance through detailed simulation studies.

% TODO briefly mention future work directions
\end{comment}
	
This work develops a shared control framework between adaptive autopilots and human operators of aerial vehicles. When human operators are faced with the control of plants having unfamiliar dynamics, they attempt to adapt their control strategy, but their performance deteriorates while the risk of loss of control increases. Adaptive control algorithms can be used to automate low-level control tasks, allowing human pilots/operators to focus on higher-level perception and decision-making tasks where their cognitive capabilities can truly be leveraged. The autonomous control algorithms presented in this thesis build on two recent advances in adaptive control theory, namely the use of closed-loop reference models for improved transient performance, and computationally efficient control designs for output-feedback systems having relative degree two or greater. 

In Chapter \ref{ch:siso_shared_ctrl} a shared control architecture between on-board human pilots and adaptive controllers with full state information was presented. In Chapter \ref{ch:mimo_shared_ctrl} the shared control architecture was developed for remote human operators and adaptive controllers having just partial state information available for feedback control. In these two settings, the human operator collaborates with the autopilot in the detection and diagnosis of the anomaly and relegates corrective actions to an adaptive flight control system. The targeted role of the human operator is motivated by the limitations that come with manual control of unfamiliar dynamical systems, as well as the cognitive and perceptive capabilities unique to human operators. Under our shared control framework, the human operator and adaptive autopilot designs form a shared response to dynamical anomalies. 

The shared controllers defined in Chapters \ref{ch:siso_shared_ctrl} and \ref{ch:mimo_shared_ctrl} are applied to several scenarios relevant to flight control through numerical simulations. The shared controller with on-board human pilots is demonstrated in the context of the roll dynamics of an aircraft, in the face of two different kinds of anomalies. In both cases, the pilot's task is to perceive if there is a change in the order of the vehicle dynamics, and convey this change to the adaptive autopilot. The resulting shared control action was shown to lead to satisfactory performance through detailed simulation studies. The shared control architecture using remote human operators is demonstrated in simulations of the longitudinal dynamics of an unmanned high altitude, long endurance aircraft. It is shown how an anomaly response using the shared controller is able to avert structural failure following an anomaly which abruptly changes actuator dynamics, and restore nominal performance.

There are several directions in which this work can be extended. One area to explore is in the development of anomaly detection and diagnosis tools to aid the human operator, based on models of how humans perceive and diagnose anomalies. Demonstrating this work on hardware platforms with a human in/on the loop is another area for future work which would provide valuable insights. 
