% Summary
\chapter{Concluding Remarks} \label{ch:conclusion}
% TODO reorg this chapter
This thesis developed a shared control architecture which synthesizes the actions of adaptive autopilots and human operators of aerial vehicles. Humans faced with the control of plants having unfamiliar dynamics attempt to adapt their control strategies, but their performance deteriorates while the risk of loss of control increases. Adaptive control algorithms can automate low-level control tasks, enabling stable and consistent closed-loop dynamic behavior when the parameters of open-loop plant dynamics are uncertain. The idea of this shared control architecture is to allow human operators to focus on higher-level perception and decision-making tasks where their cognitive capabilities can be leveraged, while using adaptive autopilots for command tracking and regulation tasks. The autonomous control algorithms presented in this thesis build on two recent advances in adaptive control theory, namely the use of closed-loop reference models for improved transient performance, and computationally efficient control designs for output-feedback systems having relative degree two or greater. 

In Chapter \ref{ch:siso_shared_ctrl} a shared control architecture between on-board human pilots and adaptive controllers with full state information was presented. In Chapter \ref{ch:mimo_shared_ctrl} the shared control architecture was developed for remote human operators and adaptive controllers having just partial state information available for feedback control. In these two settings, the human operator collaborates with the autopilot in the detection and diagnosis of the anomaly and relegates corrective actions to an adaptive flight control system. The targeted role of the human operator is motivated by the limitations that come with manual control of unfamiliar dynamical systems, as well as the cognitive and perceptive capabilities unique to human operators. Under our shared control framework, the human operator and adaptive autopilot designs form a shared response to dynamical anomalies. 

The shared controllers defined in Chapters \ref{ch:siso_shared_ctrl} and \ref{ch:mimo_shared_ctrl} are applied to several scenarios relevant to flight control through numerical simulations. The shared controller with on-board human pilots is demonstrated in the context of the roll dynamics of an aircraft, in the face of two different kinds of anomalies. In both cases, the pilot's task is to perceive if there is a change in the order of the vehicle dynamics, and convey this change to the adaptive autopilot. The resulting shared control action was shown to lead to satisfactory performance through detailed simulation studies. The shared control architecture using remote human operators is demonstrated in simulations of the longitudinal dynamics of an unmanned high altitude, long endurance aircraft. It is shown how an anomaly response using the shared controller is able to avert structural failure following an anomaly which abruptly changes actuator dynamics, and restore nominal performance.

There are several directions in which this work can be extended. One area to explore is in the development of anomaly detection and diagnosis tools to aid the human operator, based on models of how humans perceive and diagnose anomalies. Demonstrating this work on hardware platforms with a human in/on the loop is another area for future work which would provide valuable insights and help to bring this work to a higher level of maturity. In summary, designing control architectures which are able to leverage the benefits of both humans and model-based algorithmic controllers may allow for both safer and more performant systems compared to what is achievable using humans or autonomous controllers alone, and this thesis has demonstrated one of many possible realizations of such a control design philosophy.
