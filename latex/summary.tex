% Summary
\chapter{Concluding Remarks} \label{sec:conclusion}
This work develops a shared control framework between adaptive autopilots and remote human operators of aerial vehicles. The autonomous control design (autopilot) builds on two recent advances in adaptive control theory, namely the use of closed-loop reference models for improved transient performance, and computationally efficient control designs for output-feedback systems having relative degree two or greater. The targeted role of the human operator is motivated by unmanned aerial platforms remotely supervised by humans (``human-on-the-loop''), and the limitations that come with remote manual control. Under our shared control framework, the human operator and adaptive autopilot designs form a shared response to dynamical anomalies. The shared control response is demonstrated in simulation on the longitudinal dynamics of an unmanned HALE VFA model. 

% TODO merge above (CPHS) with below (GNC)

When human pilots are suddenly presented with unfamiliar dynamics, they attempt to adapt their feedback gains, but performance deteriorates which increases the potential for loss of control events. Autonomous adaptive controllers present the potential to remove the human pilot from low-level control tasks and instead allow them to make high-level decisions, such as the switching of the controller structure to a suitable dimension which allows the adaptive autopilot to retain autonomous control of the vehicle. The human pilot collaborates with the autopilot in the detection and diagnosis of the anomaly and relegates corrective actions to an adaptive flight control system. As an autonomous adaptive flight control system will be able to control these low-level tasks with ease, such a shared controller may prove to be more advantageous in the face of severe anomalies.

We explore the behavior of a shared controller in the context of the roll dynamics of an aircraft, in the face of two different kinds of anomalies. In both cases, the pilot's task is to perceive if there is a change in the order of the vehicle dynamics, and convey this change to the adaptive autopilot. The resulting shared control action was shown to lead to satisfactory performance through detailed simulation studies.

To further develop the shared control architecture, more detailed perception models of the human will need to be derived and evaluated using realistic vehicle models and scenarios. Additionally, algorithms based on human perception models may be developed to aid in autonomous anomaly detection. These form the subject of future work.