% $Log: abstract.tex,v $
% Revision 1.1  93/05/14  14:56:25  starflt
% Initial revision
% 
% Revision 1.1  90/05/04  10:41:01  lwvanels
% Initial revision
% 
%
%% The text of your abstract and nothing else (other than comments) goes here.
%% It will be single-spaced and the rest of the text that is supposed to go on
%% the abstract page will be generated by the abstractpage environment.  This
%% file should be \input (not \include 'd) from cover.tex.

This thesis addresses the problem of controlling a dynamical system subject to both parametric uncertainties and the abrupt manifestation of severe changes in the dynamic architecture of the system, by proposing a shared control architecture predicated on a suitable division of responsibilities between supervisory human operators and adaptive control algorithms. Dynamical anomalies, such as the abrupt introduction of unmodeled dynamics or time delays, present difficulties in control for both human operators and autonomous model-based control algorithms. Online adjustment to reject the effect of parametric uncertainty is possible through the proper use of adaptive control, and to a certain extent is paralleled by the learning of dynamics and adaptation of control policies by humans operators. Changes to the dynamic structure of the system, however, may lead to poor closed-loop performance and instability, regardless of whether the control loop is being closed by a human or adaptive control algorithm.

This thesis introduces a shared decision-making and control architecture based on adaptive control which tasks supervisory human operators with several high-level responsibilities in the mitigation of dynamical anomalies, to enable the recovery of closed-loop system stability and command tracking performance without transferring low-level control responsibilities from adaptive control algorithms to the human. This shared controller is defined in this thesis for systems where the full state is available for feedback as well as the case where only certain outputs are available for feedback. Anomaly response using this shared control architecture is demonstrated in several scenarios related to flight control, including the operation of an unmanned aerial vehicle whose actuator dynamics abruptly change from first-order to second-order. 