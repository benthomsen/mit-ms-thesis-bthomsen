% Shared Control
\chapter{Shared Control with Remote Human Pilot and Output Feedback}  \label{ch:mimo_shared_ctrl}
We introduce a shared decision-making and control framework with the goal of enabling the safe operation of HALE UAVs subject to both parametric uncertainties and the sudden introduction of anomalous, unmodeled dynamics as described in the preceding section. The shared control framework is based on a combination of actions by UAV autopilots and remote human operators. MRAC autopilots and complementary higher-level motion planning algorithms allow for continuous autonomous operation of the UAVs under nominal conditions. Remote human operators monitor the performance of the vehicles and are trained and able to remotely pilot the vehicle in case of autopilot failure. The remote piloting of the vehicles, however, is a daunting task due to communication delays and a weakened understanding of the vehicle dynamics, state, and environment, due to the remote nature of the task. Our shared anomaly response involves the remote human operator to diagnose and correct for the dynamical anomaly without taking over manual control of the vehicle. In Section \ref{subsec:sc_adaptive}, we describe two adaptive autopilot designs which in combination with the human operator whose precise role is described subsequently in Section \ref{subsec:sc_human}, will solve the problem presented in Section \ref{sec:problem}.

%Division of responsibilities.

\section{Adaptive Output-Feedback Control}\label{subsec:sc_adaptive}
An autonomous controller is designed to track prescribed commands for plant outputs $z_p(t)$ in (\ref{eq:plant_dynamics}). The shared control framework involves separate MRAC designs for the plant (\ref{eq:plant_dynamics}) in combination with actuator dynamics (\ref{eq:first_order_act}) and (\ref{eq:second_order_act}). The control design accommodating first-order actuators is denoted the ``nominal'' control design, and excluding exceptional failures, is the controller in use by the UAV autopilot. The control design accommodating second-order actuators is a predefined ``recovery'' controller, whose use case will be defined more fully in Section \ref{subsec:sc_human}. To achieve the control goals stated in Section \ref{sec:problem}, control design consists of
\begin{enumerate}[label=(\roman*)]
	\item baseline control design using the robust servomechanism linear quadratic regulator method (RSLQR);
	\item adaptive output-feedback augmentation for parametric uncertainties in the plant.
\end{enumerate}

Control design in each case uses an augmented linear plant formulation, where the plant (\ref{eq:plant_dynamics}) is extended with the actuator dynamics -- either (\ref{eq:first_order_act}) or (\ref{eq:second_order_act}) -- as well as integrated tracking errors 
\begin{equation}
	e_z^{\mathcal{I}}(t) = \int_0^{t} \big( z_p(\tau) - z_{cmd}(\tau)\big) d\tau.
\end{equation}

The augmented plant model with $x = \begin{bmatrix} x_p^T & x_{act}^T & (e_z^{\mathcal{I}})^T\end{bmatrix}^T$ can written compactly as
\begin{equation}
\begin{array}{c}
\dot{x}= \left(A+B_{1}\Psi_{1}^{T}+B_{r}\Psi_{r}^{T}\right) x+B_{r}\Lambda u+B_{z}z_{cmd}\\
y=Cx,\qquad z=C_{z}x
\end{array} \label{eq:augmented_plant}
\end{equation}
where $x\in\mathbb{R}^{n}$, $u\in\mathbb{R}^{m}$, $y\in\mathbb{R}^{p}$ are redefined states, inputs and outputs, respectively. 
%A full justification for this form of the plant can be found in \cite{qu2016phd, qu2016adaptive}. 
This plant contains unknown matrices $\Psi_1$, $\Psi_r$, and $\Lambda$, which hold the state-dependent plant uncertainties, state-dependent actuator dependencies, and actuator effectiveness, respectively. The exact forms of $B_r$ and $\Psi_r$ depend on whether the actuators are first-order (\ref{eq:first_order_act}) or second-order (\ref{eq:second_order_act}), and the subscript $r$ indicates the relative degree of the augmented plant. It is noted that the augmented plant model which arises from the inclusion of actuator model (\ref{eq:first_order_act}) in the plant (\ref{eq:plant_dynamics}) has relative degree two, while the augmented plant model associated with the inclusion of actuator model (\ref{eq:second_order_act}) has relative degree three. 

For control design, closed-loop reference models (\cite{gibson2013adaptive}) are designed as
\begin{equation}
\dot{x}_m = A_m x_m + B_z z_{cmd} + L e_y + \mathcal{F}_r(t), \quad y_m = C x_m
\end{equation}
where $e_y = y - y_m$, $A_m = A - B_r K^T$ with $K\in\mathbb{R}^{n\times m}$ is a baseline feedback control gain designed for the system without uncertainty using RSLQR, as described by \cite{lavretsky2013robust}. $L$ is a Luenberger-like feedback gain, and $\mathcal{F}_r(t)$ is a function used when $r \geq 2$ to recover stability properties in the presence of uncertainty. 

In what follows, control designs for the ``nominal'' controller and for the ``recovery'' controller are summarized, assuming (for simplicity, but without loss of generality) that the augmented plant model (\ref{eq:augmented_plant}) is square (i.e. $m = p$). Readers are referred to \cite{qu2016phd} and \cite{qu2016adaptive} for a more thorough treatment of the adaptive control designs in this article. 
 
\subsection{Nominal Adaptive Control Design}
%Relative degree two MIMO adaptive control.
% need a11, a10, B_1^a, Cbar, S, Rinv, epsilon, L, two tuner laws, u, F
The control design for the plant with first-order actuator dynamics summarized by giving definitions for CRM residual gain matrix $L$, function $\mathcal{F}_2(t)$, control law $u(t)$, and parameter adaptation. Note that $B_2$ represents $B_r$ from (\ref{eq:augmented_plant}) for this relative degree two plant. 

The feedback matrix $L$ is designed as follows. We define the ``relative degree one input path''
\begin{equation}
B_1^a = \alpha_0 B_2 + \alpha_1 A B_2 \label{eq:rd2-b1a}
\end{equation}
where $\alpha_i > 0$ are free design parameters. We then define
\begin{align}
S &= (C B_1^a)^T \label{eq:S}\\	\overline{C} & = S C\\ R^{-1} &= (\overline{C} B_1^a)^{-1} \big[ \overline{C} A B_1^a + (\overline{C} A B_1^a)^T\big] (\overline{C} B_1^a)^{-1} + \epsilon I \\ L & = B_1^a R^{-1} S \label{eq:L}
\end{align}
where $\epsilon > 0$ \cite[Eq. 30]{qu2015adaptive} is chosen to guarantee stability of the adaptive system. 

The function $\mathcal{F}_2(t)$ makes use of scaled error signal
\begin{equation}
	e_{sy}(t) = R^{-1} S e_y(t) \label{eq:esy}
\end{equation}
and a filtered version of this signal, $\overline{e}_{sy}(t)$, given in the form of a differential equation as
\begin{equation}
(\alpha_0 + \alpha_1 \frac{d}{dt}) \big\{ \overline{e}_{sy}(t) \big\} = \alpha_1 e_{sy}(t). \label{eq:e_sy_bar}
\end{equation}
It is worth noting that this can be represented in the Laplace $s$-domain as
\begin{equation*}
	\overline{E}_{sy}(s) = \frac{\alpha_1}{\alpha_1 s + \alpha_0} E_{sy}(s).
\end{equation*}

$\mathcal{F}_2(t)$ is then defined as
\begin{equation}
\mathcal{F}_2(t) = B_2 (\alpha_0 + \alpha_1 \frac{d}{dt})\big\{ \hat{\Psi}_m^T (t) \bar{e}_{sy}(t) \big\} \label{eq:F2}
\end{equation}
where $\hat{\Psi}_m(t)$ is a matrix of adaptive parameters. Similar to (\ref{eq:e_sy_bar}), we define filtered reference model state, $\overline{x}_m(t)$, with the differential equation
\begin{equation}
(\alpha_0 + \alpha_1 \frac{d}{dt}) \big\{ \overline{x}_{m}(t) \big\} = \alpha_1 x_{m}(t). \label{eq:xm_bar}
\end{equation}

We define a regressor vector of known signals as
\begin{equation}
\mathcal{X}(t) = \big[ (K^T \overline{x}_m)^T,\quad x_m^T,\quad \overline{x}_m^T \big]^T.
\end{equation}

The control law, $u(t)$, is then given by
\begin{equation}
u(t) = - (\alpha_0 + \alpha_1 \frac{d}{dt}) \big \{ \hat{\Psi}_{\Lambda}^T (t) \mathcal{X}(t) \big\} \label{eq:u_rd2}	
\end{equation}
where $\hat{\Psi}_{\Lambda}(t)$ is a matrix of adaptive parameters. The laws for adaptation of parameter matrices $\hat{\Psi}_m(t)$ and $\hat{\Psi}_{\Lambda}(t)$ are given by
\begin{equation}
\begin{aligned}
	\dot{\hat{\Psi}}_m(t) &= \Gamma_{m} \overline{e}_{sy}(t) e_y^T(t) S^T \\
	\dot{\hat{\Psi}}_{\Lambda}(t) &= -\Gamma_{\Lambda} \mathcal{X}(t) e_y^T (t) S^T
\end{aligned} \label{eq:rd2-adaptation}
\end{equation}
with diagonal adaptation gains $\Gamma_{m}, \;\Gamma_{\Lambda} > 0$. We note that the derivatives of the adaptive parameters, computed in (\ref{eq:rd2-adaptation}), are used to implement (\ref{eq:F2}) and (\ref{eq:u_rd2}) with the product rule of differentiation.

\subsection{Recovery Adaptive Control Design}
Control design with the second-order actuator model is similar to that described above, but requires modifications to ensure strict positive realness of the transfer matrix of the model-following error dynamics. 

The definition of $L$ is modified by replacing $B_1^a$ in (\ref{eq:rd2-b1a}) with
\begin{equation}
B_1^a = \alpha_0 B_3 + \alpha_1 A B_3 + \alpha_2 A^2 B_3 \label{eq:rd3-b1a}
\end{equation}
and proceeding with (\ref{eq:S})--(\ref{eq:L}). A definition for $\epsilon>0$ in this case can be found in \cite{qu2016phd}. To simplify notation, the operator $\Pi \{\cdot \}$ is defined as
\begin{equation}
\Pi \{ \cdot \} = \big( \alpha_0 + \alpha_1 \frac{d}{dt} + \alpha_2 \frac{d^2}{dt^2} \big) \{ \cdot \}.
\end{equation}

The function $\mathcal{F}_3(t)$ utilizes filtered error vectors $\overline{e}_{sy}^{[1]}(t)$, $\overline{e}_{sy}^{[2]}(t)$, and $\overline{e}_{sy}^{[1][2]}(t)$, defined by the differential equations
\begin{equation}
\begin{aligned} 
	\Pi \big \{ \overline{e}_{sy}^{[1]}(t) \big \} & = (\alpha_1 + \alpha_2 \frac{d}{dt}) \big \{ e_{sy}(t) \big \} \\
	\Pi \big \{ \overline{e}_{sy}^{[2]}(t) \big \} & = \alpha_2 e_{sy}(t) \\
	\Pi \big \{ \overline{e}_{sy}^{[1][2]}(t) \big \} & = (\alpha_2 \frac{d}{dt}) \big \{ \hat{\phi}_1^T(t) \bar{e}_{sy}^{[1]}(t) \big \}
\end{aligned} \label{eq:esy_rd3}
\end{equation}
where $e_{sy}(t)$ was defined in (\ref{eq:esy}), $\hat{\phi}_1(t)$ is a vector of adaptive parameters, and coefficients $\alpha_i > 0$ are free design parameters. We define the integrated and scaled measurement output error, 
\begin{equation}
e_{y}^{\mathcal{I}}(t) = \int_0^{t} L\big (y(\tau) - y_m(\tau)\big) d\tau
\end{equation}
which is used to define filtered error signals $\overline{e}_{\mathcal{I}y}^{[1]} (t)$ and $\overline{e}_{\mathcal{I}y}^{[1][2]} (t)$, given by
\begin{equation}
\begin{aligned}
	\Pi \big \{  \overline{e}_{\mathcal{I}y}^{[1]} (t) \big \} &= (\alpha_1 \frac{d}{dt} + \alpha_2 \frac{d^2}{dt^2}) \big \{ \hat{\Phi}_1^T (t) e_{y}^{\mathcal{I}}(t) \big \} \\
	\Pi \big \{  \overline{e}_{\mathcal{I}y}^{[1][2]} (t) \big \} &= (\alpha_2 \frac{d}{dt}) \big \{ \hat{\Lambda}(t) \overline{e}_{\mathcal{I}y}^{[1]} (t) \big \}
\end{aligned} \label{eq:eIy_rd3}
\end{equation}
where $\hat{\Phi}_1(t)$ and $\hat{\Lambda}(t)$ are matrices of adaptive parameters. We define operators
\begin{equation}
\begin{aligned}
	f_a \{ \cdot \} &= \big(\alpha_0 \alpha_2 B_3 + (\alpha_1 B_3 + \alpha_2 A B_3)\frac{d}{dt} \big) \{ \cdot \} \\
	f_b \{ \cdot \} &= \alpha_2 B_3 \Pi \{ \cdot \}
\end{aligned}
\end{equation}
and use these to define
\begin{multline}
	\mathcal{F}_3(t) = f_a \big \{ \hat{\phi}_1^T(t) \overline{e}_{sy}^{[1]}(t) - \hat{\Lambda}^T(t) \overline{e}_{\mathcal{I}y}^{[1]} (t) \big \} \\
	+ f_b \big \{ \hat{\phi}_1^T(t) \big[\overline{e}_{sy}^{[1][2]}(t) -  \overline{e}_{\mathcal{I}y}^{[1][2]} (t) \big ] + \hat{\phi}_2^T (t) \overline{e}_{sy}^{[2]}(t) \big \} \label{eq:F3}
\end{multline}
where $\hat{\phi}_2(t)$ is an additional vector of adaptive parameters. 

We define filtered reference model states $\overline{x}_m^{[1]}$ and $\overline{x}_m^{[2]}$ as
\begin{equation}
\begin{aligned}
	\Pi \big \{  \overline{x}_{m}^{[1]} (t) \big \} &= (\alpha_1 + \alpha_2 \frac{d}{dt}) \big \{ x_m (t) \big \} \\
	\Pi \big \{  \overline{x}_{m}^{[2]}(t) \big \} &= \alpha_2 x_m (t).
\end{aligned}	
\end{equation}

Variable $\overline{v}_m(t)$ is introduced, with artificial time derivatives, such that
\begin{equation}
\begin{gathered}
	\overline{v}_m = x_m, \quad \frac{d }{dt}\{ \overline{v}_m \} = A x_m + B_z z_{cmd} \\
	\frac{d^2}{dt^2} \{ \overline{v}_m \} = A^2 x_m + A B_z z_{cmd} + B_z \frac{dz_{cmd}}{dt} - A L e_y.
\end{gathered}
\end{equation}

The regressor vector $\mathcal{X}(t)$ is redefined as
\begin{equation}
\mathcal{X}(t) = \big[ (K^T \overline{x}_m^{[2]})^T,\quad \overline{v}_m^T,\quad \overline{x}_m^{[1]T},\quad \overline{x}_m^{[2]T} \big]^T.
\end{equation}

The control law $u(t)$ for the ``recovery'' controller is
\begin{equation}
\begin{aligned}
	u (t) = -&\Pi \big \{ \hat{\Psi}^T(t) \mathcal{X}(t) \big \} \\ - & (\alpha_1 \frac{d}{dt} + \alpha_2 \frac{d^2}{dt^2}) \big \{ \hat{\Phi}_1^T(t) \big \} e_y^\mathcal{I} (t) 
\end{aligned} \label{eq:u_rd3}
\end{equation}
where 
\begin{equation}
\hat{\Psi}(t) = \big[ \hat{\Upsilon}^T(t),\quad \hat{\Phi}_1^T(t),\quad \hat{\Phi}_2^T(t),\quad \hat{\Phi}_3^T(t) \big]^T 
\end{equation}
is a matrix of adaptive parameters. 

In this controller, the laws for parameter adaptation use second-order tuners as in \cite{qu2016phd}. We first define a regressor vector $\nu(t)$ of filtered error signals
\begin{equation}
	\nu(t) = \begin{bmatrix}
		(\overline{e}_{\mathcal{I}y}^{[1][2]} - \overline{e}_{sy}^{[1]} - \overline{e}_{sy}^{[1][2]})^T, & \quad (-\overline{e}_{sy}^{[2]})^T, & \quad (\overline{e}_{\mathcal{I}y}^{[1]})^T
	\end{bmatrix}^T
\end{equation}
and associated matrix of adaptive parameters
\begin{equation}
\hat{\Theta}(t) = \big[ \hat{\phi}_1^T(t),\quad \hat{\phi}_2^T(t),\quad \hat{\Lambda}^T(t) \big].
\end{equation}

Inputs to the second-order tuners are calculated by integrating
\begin{equation}
\begin{aligned}
\dot{\hat{\Psi}}'(t) &= \Gamma_{\Psi} \mathcal{X} e_y^T S^T \text{sgn}(\Lambda) \\
\dot{\hat{\Theta}}'(t) &= -\Gamma_{\Theta} \nu e_y^T S^T
\end{aligned}
\end{equation}
where $\Gamma_{\Psi}, \Gamma_{\Theta} > 0$ are diagonal adaptation gains. 

The desired matrices of adaptive parameters are outputs of the tuners
\begin{equation}
\begin{aligned}
	\dot{X}_{\hat{\Psi}}(t) &= \big( A_T X_{\hat{\Psi}} + B_T (\hat{\Psi}'(t))^T \big) g(\mathcal{X}, \mu_{\mathcal{X}}) \\
	\hat{\Psi}(t) &= (C_T X_{\hat{\Psi}})^T \\
	\dot{X}_{\hat{\Theta}}(t) &= \big( A_T X_{\hat{\Theta}} + B_T (\hat{\Theta}'(t))^T \big) g(\nu, \mu_{\nu}) \\
	\hat{\Theta}(t) &= (C_T X_{\hat{\Theta}})^T
\end{aligned}	
\end{equation}
where
\begin{equation}
g(\mathbf{x}, \mu) = 1 + \mu \mathbf{x}^T \mathbf{x}	
\end{equation}
is a time-varying gain with scalar gain $\mu$ described in \cite{qu2016phd}. $A_T \in \mathbb{R}^{2m \times 2m}$, $B_T \in \mathbb{R}^{2m \times m}$, and $C_T \in \mathbb{R}^{m \times 2m}$ are block diagonal matrices with diagonal blocks
\begin{equation}
A_{T,i} = \begin{bmatrix}
	0 & 1\\ -\frac{\alpha_0}{\alpha_2} & -\frac{\alpha_1}{\alpha_2}
\end{bmatrix}, \quad B_{T,i} = \begin{bmatrix}
	0 \\ \frac{\alpha_0}{\alpha_2}
\end{bmatrix}, \quad C_{T,i} = \begin{bmatrix}
	1 & 0
\end{bmatrix}
\end{equation}

Derivatives of the adaptive parameters, used in (\ref{eq:esy_rd3}), (\ref{eq:eIy_rd3}), (\ref{eq:F3}), and (\ref{eq:u_rd3}), are given by
\begin{equation}
\begin{aligned}
	\dot{\hat{\Psi}}(t) &= (C_T^\delta X_{\hat{\Psi}})^T, \qquad \ddot{\hat{\Psi}}(t) &= (C_T^{\delta\delta}X_{\hat{\Psi}})^T \\
	\dot{\hat{\Theta}}(t) &= (C_T^\delta X_{\hat{\Theta}})^T, \qquad \ddot{\hat{\Theta}}(t) &= (C_T^{\delta\delta}X_{\hat{\Theta}})^T
\end{aligned} \label{eq:rd3-adaptation-deriv}
\end{equation}
where $C_T^{\delta}, C_T^{\delta \delta} \in \mathbb{R}^{m \times 2m}$ are block diagonal matrices with diagonals $C_{T,i}^{\delta} = \begin{bmatrix} 0,~ & 1	\end{bmatrix}$ and $C_{T,i}^{\delta\delta} = -\frac{1}{\alpha_2}\begin{bmatrix} \alpha_0,~ & \alpha_1 \end{bmatrix}$. 

\section{Human Supervisor}\label{subsec:sc_human}
%Notices, reacts, instructs.
We task the remote human supervisor with the following three responsibilities for shared anomaly response on the HALE VFA platform.
\begin{enumerate}[label=\arabic*.]
	\item Timely detection and characterization of anomalous closed-loop dynamical behavior
	\item Isolation of control loop with anomalous behavior (i.e. longitudinal or lateral-directional)
	\item Commanding a change from nominal autopilot (\ref{eq:rd2-b1a})--(\ref{eq:rd2-adaptation}) to recovery autopilot (\ref{eq:rd3-b1a})--(\ref{eq:rd3-adaptation-deriv})
\end{enumerate}

The first task requires an attentive human operator able to discern that 
\begin{enumerate}[label=(\alph*)]
	\item an anomaly has occurred and control performance degradation is not caused solely by external disturbances;
	\item swift action must be taken in order to recover stability and performance;
	\item it may be possible to recover stability and performance via corrective action.
\end{enumerate}

The second task requires a human operator with knowledge and familiarity with the VFA dynamics and control structure to understand which control loop (e.g. pitch mode, roll mode, airspeed) is the source of the anomalous dynamics. 

The final task for the trained remote human operator is the transfer of this diagnosis to the autopilot, by changing the relevant controller to its ``recovery'' mode.

Note that while the remote human operator is assumed to have the training and controls necessary to disable all autopilot functionality and control the vehicle manually, this shared anomaly response deliberately circumvents any \textit{human-in-the-loop} (manual) control. 